\documentclass[12pt,letterpaper]{article}
\usepackage[utf8]{inputenx} %Codificacion del texto (ISO Latin1 encoding)

\usepackage{fancyhdr} %Permite acomodar a tu gusto la parte de arriba y
% abajo del documento
\usepackage[spanish]{babel} %Permite definir el idioma del dcumento
\usepackage{graphicx} %Permite exportar imagenes en formato eps
\usepackage{caption}
\usepackage{subcaption}
\usepackage{url} %Tipo de fuente para correos y paginas
\usepackage{pgf}
\usepackage{fleqn}
\usepackage{amssymb}
\usepackage{amsmath}
\usepackage{fancyvrb}
\usepackage{makeidx}
\usepackage{colortbl} %Permite colocar colores a las tablas
\usepackage{booktabs}
\usepackage{moreverb}
\usepackage[final]{pdfpages}
%%%%%%%%%%
%Margenes%
%%%%%%%%%%
\parskip 1mm %Espacio entre parrafos

\setlength{\topmargin}{0pt}
\topmargin      0.5cm
\oddsidemargin	0.1cm  % Ancho Letter 21,59cm
\evensidemargin 0.5cm  % Alto  Letter 27,81cm
\textwidth	17cm%15.5cm
\textheight	21.0cm
\headsep	4 mm
\parindent	0.5cm
%%%%%%%%%%%%%%%%%%%%%%
%Estilo del documento%
%%%%%%%%%%%%%%%%%%%%%%
\pagestyle{fancyplain}

\renewcommand{\footrulewidth}{0pt} %Linea de separacion inferior
\renewcommand{\headrulewidth}{0pt} %Linea de separacion inferior

\newcommand{\primaria}[1]{
	\textbf{\underline{#1}}
}

\newcommand{\foranea}[1]{
	\textbf{\textsl{#1}}
}

\newcommand{\primyfor}[1]{
	\underline{\foranea{#1}}
}

\makeatletter
\newcommand\subsubsubsection{\@startsection {paragraph}{1}{\z@}%
                                   {-3.5ex \@plus -1ex \@minus -.2ex}%
                                   {1.5ex \@plus.2ex}%
                                   {\normalfont\bfseries}}
                       
                                
                                 
\newcommand\subsubsubsubsection{\@startsection {subparagraph}{1}{\z@}%
                                   {-3.5ex \@plus -1ex \@minus -.2ex}%
                                   {1.5ex \@plus.2ex}%
                                  
                                   {\normalfont\bfseries}}
\rhead{}
\cfoot{}
\lhead{}
\newcommand{\tab}{\hspace*{4em}}
\makeatother


\begin{document}
\hfill Sociedad de Jóvenes

\hfill San Jorge \#546

\hfill El Belloto, Quilpué

\hfill Fono +56 32 233 2578

\hfill \textbf{Informe al Consistorio}
\\
\textbf{Sociedad de Jóvenes} \\ IPNA El Renuevo \\  \today \\ Periodo: Octubre 2011 - Octubre 2012\\

Al honorable Consistorio:\\

Se les hace entrega de este informe de la Sociedad de Jóvenes de la Iglesia local, para los fines informativos que anualmente se requieren presentar a la iglesia. A continuación se presentan las informaciones solicitadas.

Por favor, notar que el año de trabajo de la Sociedad de Jóvenes va de Octubre de un año al siguiente, por lo que este informe presenta las estadísticas \textbf{del período anterior}.

\section{Integrantes del departamento}

Los integrantes del departamento del período anterior, han sido reelectos, por lo que esta información es tanto válida para el periodo octubre 2011 - octubre 2012, como para el período actual octubre 2012 - octubre 2013.

\begin{table}[!h]
\centering
\begin{tabular}{|l|l|}
\hline
\textbf{Cargo} & \textbf{Nombre} \\
\hline \hline
\textbf{Presidente} & Hno. Gustavo Silva \\ \hline
\textbf{Vice-Presidente} & Hna. Lissette Aguirre \\ \hline
\textbf{Consejero} & Rvdo. Joel Tronocoso \\ \hline
\textbf{Tesorero} & Hector Espinoza \\ \hline
\textbf{Secretario} & Victor Gonzalez \\ \hline
\end{tabular}
\end{table}

\newpage
\section{Información de asistencia a reuniones}
\subsection{Información sobre las reuniones}
\begin{table}[!h]
\centering
\begin{tabular}{|l|l|}
\hline
\textbf{Día de reunión} & Sábado \\ \hline
\textbf{Hora de reunión} & 18:00 hrs \\ \hline
\textbf{Lugar de reunión} & Templo Iglesia El Renuevo \\ \hline
\textbf{Total de reuniones en el período} & 20 \\ \hline
\textbf{Total de asistentes} & 24 \\ \hline
\end{tabular}
\end{table}

\subsection{Asistencia general}
\begin{table}[!h]
\centering
\begin{tabular}{|l|c|c|c|c|}
\hline
\textbf{Nombre} & \textbf{Asistencia} & \textbf{Total Obligatorio} & \textbf{Porcentaje Final} \\ \hline \hline
Gustavo Silva  & 19 & 20 & 95\% \\ \hline
Alejandro Espinoza & 16 & 20 & 80\% \\ \hline
Victor Gonzalez & 14 & 20 & 70\% \\ \hline
Daniela Vidal & 13 & 20 & 65\% \\ \hline
Thalia Muñoz  & 13 & 20 & 65\% \\ \hline
Secia de Matus & 1 & 2 & 50\% \\ \hline
Lissette Aguirre & 10 & 20 & 50\% \\ \hline
Pastor Joel Troncoso & 10 & 20 & 50\% \\ \hline
Yetro Matus & 1 & 2 &50\% \\ \hline
Paulina Allende & 9 & 20 & 45\% \\ \hline
Alberto Sanchez & 8 & 20 & 40\% \\ \hline
Danilo Lopez  & 8 & 20 & 40\% \\ \hline
Tamara Vergara & 7 & 20 & 35\% \\ \hline
Abigail Vergara  & 6 & 20 & 30\% \\ \hline
Josue Muñoz & 6 & 20 & 30\% \\ \hline
Belen Troncoso & 5 & 20 & 25\% \\ \hline
Sebastian Aguayo & 5 & 20 & 25\% \\ \hline
Veronica Allende & 5 & 20 & 25\% \\ \hline
Pia Olivares & 3 & 20 & 15\% \\ \hline
Andrea Carcamo & 2 & 20 & 10\% \\ \hline
Gonzalo Sanchez & 2 & 20 & 10\% \\ \hline
Tomas Lopez & 2 & 20 & 10\% \\ \hline
Monica Allende & 1 & 20 & 5\% \\ \hline
Roxana Allende & 1 & 20 & 5\% \\ \hline
\end{tabular}
\end{table}

\section{Planes y objetivos}
\begin{table}[!h]
\begin{tabular}{l c}
\textbf{1. Planificación estructurada de estudios durante el año.} & \textbf{Logrado} \\
\tab a. Revisiones mensuales. \\
\tab b. Marzo-abril: Dones ministeriales.\\
\tab c. Expositores de temas fijos. \\
\tab \tab - Pastor Joel.\\
\tab \tab - Hno. Alberto Sanchez.\\
\textbf{2. Realizar trabajos evangelisticos.} & \textbf{No logrado} \\
\tab a. Repartición de tratados. \\
\tab b. Invitación a la reuniones. \\
\tab c. Realización de visitas.\\
\textbf{3. Solidificar las relaciones entre los integrantes de la Sociedad.} & \textbf{Logrado}\\
\tab a. Actividades extra-programáticas.
\end{tabular}
\end{table}

\section{Actividades relevantes}

La mayoría de las actividades realizadas fueron de caracter recreativo, a excepción de las reuniones distritales. Algunas de estas actividades se destacan a continuación:

\begin{description}
\item[Paseo a Cerro La Campana:] (31 de octubre 2011) Se realiza paseo con distintos integrantes de la Sociedad, en conjunto con el Pastor Joel Troncoso. No se registran incidentes.
\item[Asado partido de Chile] (11 de noviembre 2011) Con el motivo de un partido de Chile, se realiza  un asado en la casa del Hno. Alberto Sanchez.
\item[Malón Navideño] (30 de diciembre 2011) Se realiza una gala en la casa del Hno. Alberto, en la cual los asistentes debían ir con tenidas formales, lo cual se complementó con la entrega de regalos del amigo secreto.
\item[Encuentros Distritales] \textit{12 de mayo del 2012} en la IPNA de Calera. \textit{9 de junio del 2012} en la iglesia local. \textit{8 de septiembre del 2012} en la IPNA de Limache.
\item[JUPNA La Redención] (12-15 de octubre del 2012) Asisten 12 miembros de la Sociedad, mas Alberto Sanchez (ex-secretario JUPNA) y Sebastian Aguayo (ex-presidente JUPNA). Se destaca la primera participación en este evento, de las gemelas Allende.
\end{description}

\newpage
\section{Informe financiero}
A continuación presentamos el detalle contable de la Sociedad de Jóvenes en el período octubre 2011 - octubre 2012:
\begin{table}[!h]
\centering
\begin{tabular}{|l|c|c||c|c|c|}
\hline
& \textbf{Ingresos} & \textbf{Egresos} & \textbf{Ofrendas} & \textbf{Ventas} & \textbf{Otros}\\ \hline \hline
oct-11 & \$ 22.080 	& \$ 2.000 & \$ 4.080 & \$ & \$ 18.000 (*) \\ \hline
nov-11 & \$ 930 	 &\$ - & \$ 930 & \$ & \$ \\ \hline
dic-11 & \$ 3.840 	 &\$ - & \$ 3.840 & \$ & \$ \\ \hline  
ene-12 & \$ -   	 &\$ -  & \$ - & \$ & \$ \\ \hline
feb-12 & \$ -   	 &\$ -   & \$ - & \$ & \$ \\ \hline
mar-12 & \$ 3.985 	 &\$ - & \$ 3.985 & \$ & \$ \\ \hline  
abr-12 & \$ 2.190 	 &\$ -   & \$ 2.190 & \$ & \$ \\ \hline
may-12 & \$ 3.660 	 &\$ -   & \$ 3.660 & \$ & \$ \\ \hline
jun-12 & \$ 2.460 	 &\$ 2.000 & \$ 2.460 & \$ & \$ \\ \hline
jul-12 & \$ 23.750 	 &\$ 5.160 & \$ 1.390 & \$ 22.360 & \$ \\ \hline
ago-12 & \$ 3.220 	 &\$ 5.160 & \$ 2.770 & \$ & \$ 450 \\ \hline
sep-12 & \$ 50.010 	 &\$ 25.796 & \$ 3.010 & \$ 47.000 & \$ \\ \hline
oct-12 & \$ -   	 &\$ 43.500 & \$ - & \$ & \$ \\ \hline
\textbf{Subtotal} & \$ \textbf{116.125} & \$ \textbf{83.616} & \$ 28.315 & \$ & \$  \\ \hline
\end{tabular}
\end{table}

\begin{table}[!h]
\centering
\begin{tabular}{|l|c|}
\hline
\textbf{Saldo directiva anterior} (*) & \$ 18.000 \\ \hline
\textbf{Saldo final} &  \$ 32.509 \\ \hline
\end{tabular}
\end{table}

\section{Evaluación del período}
En general, el período se evalúa de muy buena manera. Se logró establecer un mecanismo muy formal de trabajo en la directiva, la cual se vió reflejada en la manera en que se organizaron las reuniones durante todo el año: si no se preparaba un tema para la reunión, no se realizaba la reunión. Esto, gracias a que tuvimos muy claro, que el propósito de nuestros encuentros en la iglesia, es para aprender del Señor, y no para simplemente reunirse con amigos.

Se buscó mucho la unión entre los hermanos de la Sociedad, y esto podemos verlo reflejado actualmente. Existe mucha empatía entre los hermanos.

\subsection{Aspectos positivos del trabajo}
Se establecieron criterios y se definieron conceptos para las reuniones que han permitido hacer que las estas sean de excelencia para el Señor. Nuestra prioridad siempre fué alabar al Señor en toda actividad, y es por lo mismo que se buscó mucho la participación y la motivación de los jóvenes asistentes a la Sociedad. Gracias a esto, existe un ambiente familiar y de amigable, el cual invita a todos los jóvenes a que se motiven a participar de nuestras reuniones.

\subsection{Metas para el próximo período}
Existen muchas metas, tales como evangelizar, trabajar en la obra que se está levantando en Valparaíso, \textit{apoyar} el trabajo con los niños que hace la Hna. Margarita Berríos, pero principalmente está la JUPNA.

Tenemos como meta entregar una JUPNA gratuita (o lo más económica posible), y que genere un impácto en la vida de los asistentes, disponiendo de una organización armoniosa y agradable, que permita integrar a todos los jóvenes de la IPNA. Somos ambiciosos en ese sentido, y por lo mismo, estamos ansiosos de recibir el apoyo de la iglesia local, para tan ardua tarea.

\section{Fortalezas y/o debilidades}
La mayoría de los jóvenes que asisten, son estudiantes universitarios (a diferencia de lo que ocurría antes, que eran en su mayoría escolares), por lo cual, en fechas críticas tales como los fines de semestre, la asistencia y capacidad de compromiso de los miembros, tanto como de la directiva, se ve fuertemente afectada.

Es por lo mismo que la asistencia se ve altamente mermada en épocas complicadas, pero estamos confiados en el Señor, que el compromiso de los jóvenes, es más fuerte, y eso se refleja en las asistencias una vez que esos períodos culminan.

\section{Palabras finales}
Quisieramos expresamente darle las gracias en primer lugar a nuestro Señor Jesucristo, el cual ha permitido llegar hasta esta instancia y nos permite reunirnos a alabar su nombre. Y en segundo lugar, queremos agradecer enormemente, el trabajo que ha hecho el Pastor Joel Troncoso: él ha sido un pilar importantísimo en lo que creemos que es el éxito de este período y confiamos plenamente en que su apoyo continuo, se hará presente cada vez que necesitemos de una palabra sabia, certera y llena de amor.
\\

Esperando que el Señor les bendiga y les guarde, se despide\\

Sociedad de Jóvenes.
\end{document}