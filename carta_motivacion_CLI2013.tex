\documentclass[12pt,letterpaper]{article}
\usepackage[utf8]{inputenx} %Codificacion del texto (ISO Latin1 encoding)

\usepackage{fancyhdr} %Permite acomodar a tu gusto la parte de arriba y
% abajo del documento
\usepackage[spanish]{babel} %Permite definir el idioma del dcumento
\usepackage{graphicx} %Permite exportar imagenes en formato eps
\usepackage{caption}
\usepackage{subcaption}
\usepackage{url} %Tipo de fuente para correos y paginas
\usepackage{pgf}
\usepackage{fleqn}
\usepackage{amssymb}
\usepackage{amsmath}
\usepackage{fancyvrb}
\usepackage{makeidx}
\usepackage{colortbl} %Permite colocar colores a las tablas
\usepackage{booktabs}
\usepackage{moreverb}
\usepackage[final]{pdfpages}
%%%%%%%%%%
%Margenes%
%%%%%%%%%%
\parskip 1mm %Espacio entre parrafos

\setlength{\topmargin}{0pt}
\topmargin      0.5cm
\oddsidemargin	0.1cm  % Ancho Letter 21,59cm
\evensidemargin 0.5cm  % Alto  Letter 27,81cm
\textwidth	17cm%15.5cm
\textheight	21.0cm
\headsep	4 mm
\parindent	0.5cm
%%%%%%%%%%%%%%%%%%%%%%
%Estilo del documento%
%%%%%%%%%%%%%%%%%%%%%%
\pagestyle{fancyplain}

\renewcommand{\footrulewidth}{0pt} %Linea de separacion inferior
\renewcommand{\headrulewidth}{0pt} %Linea de separacion inferior

\newcommand{\primaria}[1]{
	\textbf{\underline{#1}}
}

\newcommand{\foranea}[1]{
	\textbf{\textsl{#1}}
}

\newcommand{\primyfor}[1]{
	\underline{\foranea{#1}}
}

\makeatletter
\newcommand\subsubsubsection{\@startsection {paragraph}{1}{\z@}%
                                   {-3.5ex \@plus -1ex \@minus -.2ex}%
                                   {1.5ex \@plus.2ex}%
                                   {\normalfont\bfseries}}
                       
                                
                                 
\newcommand\subsubsubsubsection{\@startsection {subparagraph}{1}{\z@}%
                                   {-3.5ex \@plus -1ex \@minus -.2ex}%
                                   {1.5ex \@plus.2ex}%
                                  
                                   {\normalfont\bfseries}}
\rhead{}
\cfoot{}
\lhead{}
\newcommand{\tab}{\hspace*{4em}}
\makeatother


\begin{document}

$ $ \\ \textbf{Asunto: Apoyo de Participación en CLI 2013} \\ A: Departamento de Informática UTFSM \\ Valparaíso, Chile \\  \today \\

Estimados,\\

Se les hace llegar este documento a modo personal, con el fin de buscar financiamiento para participar en el 6º Congreso Latinoamericano de Ingeniería(CLI), el cual se desarrollará el 4,5 y 6 de octubre del presente en la ciudad de Buenos Aires, Argentina.\\

Dado que el año 2011 participé de este Congreso (donde también se recibió apoyo económico del Departamento de Informática), en donde aprendí el rol del ingeniero como parte de la sociedad, y la discusión de temas tan importantes tales como la autosustentabilidad de las energías y la importancia de contar con ingenieros comprometidos socialmente, los cuales buscan fortalecer la integración productiva, económica y social de los pueblos latinoamericanos.\\

Busco participar nuevamente de este Congreso en su versión año 2013, con el fín de aprender y profundizar más aún de los temas que se tratarán en este congreso, los cuales son:\textit{``Modelos de desarrollo industrial y la innovación productiva",``Soberanía energética y uso responsable de recursos naturales"} y \textit{``Integración y desarrollo de las comunicaciones"}; principalmente motivado por la excelente experiencia que fue el CLI 2011.

Dado que la Federación de Estudiantes de la UTFSM (FEUTFSM), está organizando a la delegación chilena, se han dispuesto de buses para el traslado de ida y vuelta del CLI. El costo del pasaje en total es \$75,000 pesos chilenos (CLP). Adicional a esto, se debe pagar el costo de inscripción del CLI, el cual es de \$300 pesos argentinos, aproximadamente CLP\$27,000.\\

Mediante el presente, les hago solicitud de CLP\$102,000 (ciento dos mil pesos chilenos), para cubrir los costos del viaje ida y vuelta al CLI 2013.\\

Sin nada más que agregar, se despide atentamente,\\
\textit{
$ $ \\\textbf{Victor Gonzalez Rodriguez}\\
Estudiante Ingeniería Civil Informática \\
UTFSM Casa Central
}
\end{document}