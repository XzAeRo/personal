\documentclass[11pt,a4paper]{moderncv}
%\moderncvtheme[blue]{casual}
\moderncvtheme[blue]{classic}
\usepackage[utf8]{inputenc}
\newcommand{\myname}{\textbf{Cristián Maureira}}
\usepackage[scale=0.8]{geometry}
%\setlength{\hintscolumnwidth}{3cm}
%\AtBeginDocument{\setlength{\maketitlenamewidth}{6cm}}
\AtBeginDocument{\recomputelengths}

% personal data
\firstname{Victor}
\familyname{González}
\title{Ingeniero Informático Memorista}
\address{Calle Del Copado \#3170}{Los Pinos, Quilpué, Chile}
\mobile{+56 9 88972537}
\phone{+56 32 2829009}
%\fax{fax (optional)} 
\email{victor.gonzalezro@gmail.com}
%\extrainfo{asdf}
\photo[64pt]{cv}
%\quote{Curriculum Vitae}
%\nopagenumbers{}

%content
\begin{document}
\maketitle

\section{Información Personal}

\cvcomputer{Nombre Completo}{Víctor Andrés Roberto González Rodríguez}
	   {RUT}{16.970.834-7}

\cvcomputer{Fecha de Nacimiento}{27 July, 1988}
	   {Nacionalidad}{Chileno}
	   
\section{Educación}

\cventry{2007 al presente}
        {Ingeniería Informática}
        {Universidad Técnica Federico Santa María (UTFSM)}
        {Valparaíso, Chile}
        {\emph{(memorista)}}
        {}
\cventry{1994-2006}
        {Enseñanza Básica y Media}
        {Colegio Esperanza}
        {Quilpué, Valparaíso, Chile}
        {}{}

\section{Experiencia}
\subsection{Vocacional}
        
\cventry{Febrero 2014 al Presente}
        {Desarrollo de Software/Web}
        {Desarrollo de  un portal web con análisis y reportes de una base de datos nacional de variables eléctricas de todas las sucursales de Banco Estado}
        {\textbf{Keywords:} PHP, Javascript, CSS3, Apache, Linux, SSH, Python, Excel (xlrd Python module), Data Mining}
        {\textbf{Empleador:} Fundación Chile}
        {Demo: \url{http://www.gestionenergeticafch.cl/dev/}}

\cventry{Marzo 2013 a Diciembre 2013}
        {Desarrollo de Software}
        {Mantenimiento y desarrollo de soluciones web para empresas, tales como control de stock, sistemas POS y desarrollo de soluciones web}
        {\textbf{Keywords:} PHP, Javascript, CSS3, Apache, Linux, SSH}
        {\textbf{Empleador:} SubRed EIRL Ltd}
        {\url{http://www.subred.cl/}}

\cventry{Mayo 2012 a Mayo 2013}
        {Ayudante en Laboratory of Interdiciplinary Research on Astro-Engineering Research}
        {Mantenimiento y desarrollo del proyecto ALMA LEGO Simulator (ALMA significa \href{http://www.alma.cl/}{Atacama Millimeter/Submillimeter Array)}. Soporte y ayuda del proyecto TLegoScope.}
        {\textbf{Keywords:} C/C++, documentación, Qt4}
        {\textbf{Empleador:} Computer Systems Research Group (CSRG), UTFSM}
        {\url{http://twiki.csrg.cl/twiki/bin/view/ACS/LegoProjectIDI}}

\cventry{Marzo 2011 a Noviembre 2012}
        {ALMA-UTFSM Ayudante de Investigación}
        {Desarrollo de software del proyecto ALMA LEGO Simulator. Desarrollo de la interface usuaria e implementación. Integración con framework de computación distribuida \emph{ALMA Common Software}}
        {\textbf{Keywords:} C/C++, Qt4, J2EE, Python, Bash, ALMA Common Software Framework}
        {\textbf{Empleador:} Computer Systems Research Group (CSRG), UTFSM}
        {\url{http://twiki.csrg.cl/twiki/bin/view/ACS/LegoProjectFsw}}
        
\cventry{Abril 2010 a Julio 2011}
        {Miembro de Proyecto}
        {Desarrollo de un una plataforma CRM privada para la Red de Ex-Alumnos USM, construida en J2EE e Hibernate para cumplir estándares académico-organizacionales}
        {\textbf{Empleador:} Red de Ex-Alumnos USM}
        {}
        {\url{http://www.exalumnos.usm.cl/}}
        
        
\subsection{Prácticas}


\cventry{Diciembre 2012 a Marzo 2013}
        {Desarrollo de Software e Investigación}
        {Mejora del framework PHP Yii para necesidades organizacionales. Implementación del proyecto OpenERP para investigación de compatibilidad de mercado. Desarrollo de soluciones web para empresas}
        {SubRed EIRL Ltd}
        {}
        {\url{http://www.subred.cl/}}
        
        
\subsection{Misceláneo}


\cventry{2009-2010}
        {Desarrollador y co-fundador}
        {BBS/IB Bienvenido a Internet. Software desarrollado en Python}
        {}
        {}
        {\url{http://www.bienvenidoainternet.org/}}
        
\section{Languajes}
\cvlanguage{Español}{Nativo}{Hablar, Leer, Escribir, Técnico y Coloquial}
\cvlanguage{Inglés}{Gold TOEIC certification}{920 puntos: International Professional Proficiency}

\section{Habilidades Computacionales}
\cvcomputer{Programación}{C/C++, Python, J2EE, C\#}
           {Paralela}{CUDA (primitivo), POSIX threads}
\cvcomputer{Sistemas Operativos}{Distribuciones Linux, Windows XP y superiores}
           {Misceláneos}{Numpy, \LaTeX}
\cvcomputer{Administración de Sistemas}{Apache, Git}
           {Desarrollo Web}{PHP, CSS3, MySQL, PostgreSQL, HTML5, Javascript (JQuery, JSON, AJAX)}

\section{Participación en Eventos}
%\cventry{2012}
%        {SFB Meeting}
%        {}
%        {}
%        {}
%        {}

\cventry{2012}
        {Quantum Information Processing}
        {Charlas cortas sobre el estado del arte en la computación cuántica}
        {Profesor Dan Marinescu, University of Central Florida}
        {Valparaíso, Chile}
        {\url{http://cci.inf.utfsm.cl/?p=399}}
        
\cventry{2012}
        {Cloud Computing}
        {Charlas cortas sobre el estado del arte en cloud computing}
        {Professor Dan Marinescu, University of Central Florida}
        {Valparaíso, Chile}
        {\url{http://cci.inf.utfsm.cl/?p=399}}

\cventry{2011}
        {ENEI-CLI}
        {Encuentro Nacional de Estudiantes de Ingeniería, Congreso Latinoamericano de Ingeniería}{}
        {Tandil, Argentina}
        {\url{http://www.cefce.com.ar/?p=3154}}

\cventry{2011}
        {Encuentro Linux}
        {Reunión nacional de usuarios chilenos de Linux}
        {Presentación sobre el proyecto ALMA LEGO Simulator y la integración con ALMA Common Software}
        {Puerto Montt, Chile}
        {\url{http://2011.encuentrolinux.cl/}}
        
\section{Honores y Premios}

\cvlistitem{Premio a la Innovación 2011 para el proyecto ALMA LEGO Simulator, otorgado por %
            ``Instituto Internacional para la Innovación de Negocios (3IE)'', UTFSM}


\end{document}
